\documentclass[]{article}

\usepackage{indentfirst}
\usepackage{cmap}					% поиск в PDF
\usepackage{datetime}				% вывод времени
%\usepackage[T2A]{fontenc}			% кодировка
%\usepackage[utf8]{inputenc}			% кодировка исходного текста
\usepackage[english,russian]{babel}	% локализация и переносы
\usepackage{amsmath, amsfonts, amssymb, amsthm, mathtools}
\usepackage{icomma}
\usepackage{mathrsfs}
\usepackage{graphicx}
\usepackage{ upgreek }
\usepackage{pdflscape}%ориентация страницы
%\usepackage{lscape}	%поворот текста
%\usepackage[pdftex]{lscape}	%поворот страниц
\usepackage{tabularx}	%хитрые таблицы
%\usepackage{xcolor}
%\usepackage{hyperref}
%рисунки и все с ними связанное
\graphicspath{{picture/}}
\setlength\fboxsep{3pt}
\setlength\fboxrule{1pt}
\usepackage{multirow}		%для столбцов объедин
\usepackage{float}

\newcommand{\tyr}{Тырныауз }
\begin{document}
	
\section{Тех. описание}	
\subsection{День 1}
Сели в заказанное такси в 7.00
Пока доехали, пока разгрузились, было уже 8.00. Пока ехали, было решено немного откорректировать маршрут- пойти не по тропам, а по дороге, и выйти на нужный трек- на горизонте появились облака, а автор отчета, по которому готовился этот участок пути указывал, что под дождем там не очень комфортно идти.

Впрочем, дождя так и не случилось, и за исключением последних ходовых часов, весь день жарило солнце. 
Дорога от тырнауза до верховьев д.р.Тырныауз достаточно широкая, с покрытием типа грунтовка, и весьма не малолюдная- за полдня на ней мы встретили около 5 машин.

Воды в низовьях долины нет совсем, поэтому если группа планирует там пойти- лучше набрать с собой побольше. Через несколько часов пути, примерно в 11.30, мы встретили мужичка травника Володю, который подсказал нам что недалеко есть родник (который указан только на Ляпинке). Перед поворотом на родник мы при сделали привал-обед в 12.00, наиболее бойкая часть команды пошла на разведку, а потом и сходила за водой для всех. 

После обеда мы подрезали планируемый трек, и пошли вдоль всяких разваленных шахт до старой морены, посмотрели на запасной вариант ночевки возле шахт, дружно решили что оно нам не надо, и пошли дальше.

Поднялись по морене, вышли на еще одну грунтовку.
К этому моменту стало чуть прохладнее, идти сталдо приятнее. В этой местности довольно обширная сеть дорог, проложенная шахтерами, но в плохой видимости она только немного путала, нежели чем помогала.
Ручей, который нарисован на картах, мы так и не смогли обнаружить, и долго искали место ночевки. К счастью, один из участников нашел два микроозера (которые тоже указаны на ляпинке, хоть и чуть дальше, чем фактитчески были). Там и заночевали в 19.00. Периодически, когда раздувало облака, мы видели разные дырки в горах- бывшие шахты. которых тут в избытке. Отличнная штука в качестве достопримечательности, особенно если послушать М.В. Расторгуева, который утверждал, что тут в нулевых мужики с автоматами бегали. 

Приятный бонус этой ночевки: виден \tyr и Баксанская долина с необычного ракурса, ловит сеть.

\subsection{День 2}
Утром вышли в 8.00, перевал видно, хоть из-за него выглядывают облачка. Движемся сначала по дороге, затем по старым моренам, идут все неплохо, и мы доходим до перевала \tyr в 10.40. На перевале мерзкое облако, совершаем все ритуальные действия (записка-фотка-шоколад), и бегом уходим оттуда. С подьемной стороны перевал не представляет никакой сложности даже в плохую погоду. Со спусковой- в целом тоже, но одна девочка все равно идет  неуверенно.
Спуск с перевала представляет собой мелкую осыпь с чуть видной тропой. В 12.10 спускаемся на зеленые поляны рядом с ручьем, обедаем. Пока идет обед,замечаю, что наш следующий перевал на сегодня окутан недружелюбным облаком, которое скоро припозает нам. Облако начинает протекать, мы ставим тенты палаток и сидим так час. Дальше становится ясно, что сегодня мы не пойдем перевал - с него все еще тянет дождевое облако, а в описании четко указано- в дождь это гадость полная. Поэтому мы переносим палатки на площадки в 50 метрах ниже, и встаем на ночевку в 16.00. Ночеки неплохие- есть подобие ветрозащитных стенок, вода недалеко.

\subsection{День 3}
Вышли с ночевки в сторону пер.Суарык  7.00, начали движение по моренам, затем вышли на приятную траву. На перевал Суарыкк вышли в 8.15. По мнению автора этого отчета, на пер.Суарык подниматься в дождь также опасно, как на любой травяой перевал 1А.
На перевале красивый вид, видно и Эльбрус, и наш предстоящий перевал Джаловчат.
Спуск с перевала начинается с едва заметной тропы по мелкой сыпухе, периодически тропа пропадает. 

В итоге мы попали в приятную зеленую долину, и к сожалению, сбились с тропы, поэтому далее просто траверсировали склоны  слева ручья ПХД держась примерно трека, но на 50м выше.
Мы шли то по травяным полям, то по козьим тропам, то форсировали приятный хвойный лес. Когда мы оказались над местом где надо переходить реку, начали спускаться вниз, и все таки увидели тропу.
Перешли ручей, и начали двигаться по правому борту ручья ПХД. 

Вышли в д.р. Кыртык на знакомую дорогу. За пару часов добежали до Верхнего Баксана (вышли в 13.30), где у нас  была заброска. К сожалению, из-за некоторого недопонимания между руководителем и ответственным за заброску, сделали лишний крюк.

После потрошения заброски пошли в долину Адырсу прихватив с собой пса Месснера. 
Долина Адырсу хорошо хожена, поэтому в тщательном описании не нуждается. 
В 17.00 дошли до пограничников, после соблюдения всех формальностей встали на ночевке сразу за ними- на хорошо оборудованной площадке.

К этому моменту мы отставали от графика на 1-2 перехода.
	
	
\subsection{День 4}	
В этот день нам предстоял подъем до ночевок под пер.Джаловчат. Двигались от м.н. до д.р.Джаловчат по хорошей дороге. Далее двигались по набитой тропе в д.р.Джаловчат. Преодолев несколько ступеней,  В 13.30 пришли на хорошие ночевки в бывшем озере, не дойдя до планируемых 1 переход. Жаркое солнце немного вымотала людей, поэтому решили встать на ночевке прямо тут.
	
	
\subsection{День 5}
Вышли с ночевки в 6.35, и начали двигаться слегка забирая вправо вверх ПХД по старой морене ледника. За 40 минут дошли до планируемого места ночевки. Там становится ясно, что год очень малоснежный- снежник, который обычно идет до большого камня перед подьемом на Джаловчат почти весь вытаял. Далее идем плотной группой. За 1 маленький переход доходим до начала основного подьема на Джаловчат. После отдыха начинаем подниматься выше по некрутым скалам, выбирая наиболее безопасный путь. Спустя 20 минут движения выходим на турики, впрочем, быстро их теряем, и двигаясь по курумнику, доходим до пер.Джаловчат.


Начинаем надевать системы и кошки, вешаем одну полную веревку вниз. На леднике Курмычи снега почти нет, а там где он есть, он очень не глубокий. Дюльферяем вниз по очереди, после дюльфера все идут на личной технике вниз до небольшого островка с камнями, где мы и собираемся. Снежная обстановка позволяет не связываться. 
Последним уходит Чаплыгин, и выманивает за собой собаку- та отказалась ввязываться в импровизированную систему, и бегала на вольном выпасе.
В итоге героический пес спустился на личной технике, и довольный убежал к основной части команды.
Всего дюльфер занял 2 часа. 

Далее за 40 минут дошли до планируемого места ночевки нагнав график. Сначала шли по открытому леднику, затем по камушкам и снегу. Ночевали в уютном кармане под перевалом. 

\subsection{День 6}
Утром проснулись в 4.00, увидели что на улице очень плохая погода. и решили отложить подьем до 8.00. Чуда не случилось, в итоге мы отсиживались весь день, то играя в 'Нечто', то сжигая всякие горелки.

\subsection{День 7}
Опять встаем в 4 утра, погода хороша, ппоэтому начинаем собираться. В 6.20 выходим, идем по средней и крупной сыпухе,  в 6.42 приходим на пер.Курмычи. Одну девочку горнит, помимо этого вся группа была уставшей после внезапной 'дневки', и мы решаем пойти запасным путем, минуя вершины сразу вниз.

Спускались по мелкой осыпи, которая сначала была приятно подморожена. Шли плотнй группой, сыпоха хоть и мелкая, но когда начинает ехать, тащит средние камни. Кажется что есть тропа, но возможно это плод фантазий. В итоге в 9 часов доходим до травы, и дальше идем по знакомой тропе. Спуск вниз сложностей не представляет, и в 11 часов мы выходим к началу прижимов в долине Адылсу. Там приваливаемся, и дальше парами доходим до поста пограничников- несмотря на индивидуальные пропуска, выпускать людей без руководителя дежурный отказался.

В 12.00 мы были в лагере Джантуган. Поляна рядом с очагом забита полностью, поэтому ночуем на левой стороне реги (орогр.).


\end{document}