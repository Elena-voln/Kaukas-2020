


\documentclass[]{article}

\usepackage{indentfirst}
\usepackage{cmap}					% поиск в PDF
%\usepackage[T2A]{fontenc}			% кодировка
%\usepackage[utf8]{inputenc}			% кодировка исходного текста
\usepackage[english,russian]{babel}	% локализация и переносы
\usepackage{amsmath, amsfonts, amssymb, amsthm, mathtools}
\usepackage{icomma}
\usepackage{mathrsfs}
\usepackage{graphicx}
\usepackage{ upgreek }
\usepackage{array}
\usepackage{longtable}

%рисунки и все с ними связанное
\graphicspath{{pic/}}
\setlength\fboxsep{3pt}
\setlength\fboxrule{1pt}

\newcommand{\tyr}{Тырныауз }

\begin{document}
	
\textit{О мой милый читатель, к сожалению в этой повести, наполненной и весельем, и трогательными моментами, будет встречаться мое нытье. Потому что куда, как ни на бумагу, лучше излить душевные страдания. Если тебя они очень утомляют, пропускай написанное курсивом.   
}	


\section{Введение}


Вся эта история началась давным-давно, в те прекрасные времена, когда ты, дорогой читатель, еще не погрузился в атмосферу тотального заточения, когда маску ты носил только в метель в каких - нибудь горах.

Бесснежным декабрем девятнадцатого, три фигуры, сидя в антикафе ”Белый лист”, обсуждали планы на прием и обработку новобранцев. Итог собрания был таков- ШГТ быть( пусть и руководов всего четыре, бывало и меньше). Стартовала агиткампания,и она вполне себе удалась- всего в школу было записано 139 человек. Мы с Мотренко довольно быстро набрали отделения, я отстрелила пару неадекватов, за что получила ачивку “фем-нацистка”.

Своим чередом шли первые выходы ШГТ, все было отлично- люди в команде преимущественно адекватные, здоровые, и вообще, мечта, а не новички. 

Грянул вирус. В тартарары полетели не только очные лекции, занятия и тренировки, но и майский поход- он планировался для формального получения опыта для большей части команды. Пока я долго и муторно общалась с МКК на тему выпуска людей без опыта в двойку, ребята времени не теряли(нет) - учили узлы и заказывали пропуска.\textit{В итоге один пропуск мы забирали у Бориса, а другой был на руках за день до отлета.}  

В июне, после послаблений в  цифровом зиндане, мы начали активно и много тренироваться- нам предстояло за месяц научиться делать то, что в обычном режиме занимает апрель, май и июнь. Но ничего, подумала я, ребята ответственные, и понимают что горы это не шутки, и мы должны уметь выбираться из самых тяжелых передряг самостоятельно, даже если они возникли внезапно. Я проводила довольно много тренировок, с пониманием относясь к тому, что у части людей из-за короны работа/учеба по выходным, а поэтому на каждой тренировке мне приходилось жонглировать своим временем так, чтоб успеть объяснить кучу технических приемов людям с разной подготовкой, и при этом успевать следить за их выполнением, и хорошо бы еще чтоб никто не тупил. 

\textit{“Неблагодарные ***** ! *** я их куда поведу, ****** !”- кричала я дома Чапе. Дело было вот в чем: видимо после пары тренировок, ребятам надоело, и они начали прогуливать. В итоге, перед последней, заключительной тренировкой, я рассчитывала что будут все, кроме нашего медика (между прочим, совершающего подвиги в период пандемии). Но нет, почему-то вся толпа решила, что они охуенно знают всю технику. Я услышала кучу отмаз- от празднования дня рождения мужа с родней, до ”жесткого джетлага после прилета”.Ну хоть спасибо что не “собака снарягу съела” . В тот момент я еще играла в “доброго руководика”, и смотрел на это мой друг и наставник Леша Чаплыгин, сочувственно- наверное, многие через это проходили. Через севшую на шею команду, я имею ввиду.}


Что же, закончились тренировки, наши локальные часы показывают начало июля, две недели до вылета. Окончательно сформировался состав команды \textit{привет Никите Д., который сказал что выписывается из похода меньше чем за сутки до выпуска после моего звонка}, пора бы и вас познакомить с ними. 	

В целом, команда отличная, почти все как на подбор- лоси. Не особо активные в плане обязанностей, но все равно неплохие ребята. 

\textit{Стоит сказать пару слов про выполнение походных обязанностей. Один умный товарищ решил после похода решил взять дополнительный отпуск, и покататься на “мототайке”. Видимо из-за этого, времени перед походом осталось мало, и у него не вышло сделать нам раскладку. Точнее, ее подобие было составлено за 8 часов до сбора команды на закупку и запаковку, и в итоге, и правки в раскладку, и распределение продуктов по людям, и координирование людей на закупке повисло на мне. И все бы ничего, если бы чудесный завхоз паралельно с этим не ебал мне мозг соотношением КБЖУ. Типа, а чего ты так поправила, там теперь белков больше, чем надо. Нет, мне нравится заниматься табличками, я бы с удовольствием потратила пару вечеров чтобы объяснить человеку как это надо делать, но обратились ко мне в 8 утра в день запаковки, к сожалению. В итоге в последнюю неделю перед вылетом, вместо работы и своих дел (например, завснарских и руководских) я занималась завхозничеством. Завхозу слава!!!}


Перейдем же из этого затянувшегося введения в основную часть нашего приключения.

	
\section{День -3, отъезд}
Путешествие началось с Павелецкого вокзала. Большая часть команды, за исключением трех человек летела одним рейсом, поэтому мы договорились поехать на одном аэроэкспрессе- и спокойнее, и веселее. Все, кроме нашего талантливого завхоза уже сидели или прощалтись с близкими в модной электричке. Минута до отправления, и надежды на то, что наш завхоз успеет, таяли как мороженное в жаркий летний день.
По итогу, завхоз опоздал и на аэроэкспресс, и на самолет. Спасибо нашему координатору Сереге Нилову, который из Москвы обзванивал такси в Беслане (куда почему-то решил прилететь опоздавший), с целью найти машину нашему товарищу.
По итогу, поймали мы Антона в ночь на повороте в Адылсу, куда он удачно попал за 7 минут до нас.

В машине я толкаю речь, про то что вести себя надо хорошо, если что-то болит- сообщить мне и медику, поляну Ахии засирать не надо и прочие стандартные вещи. Погранзаставу проскакиваем быстро- теперь у погранцов есть ноутбуки, и даже старые записи оцифрованы. Они тщательно записывают, куда мы планируем ходить, Гавля смешит ребят рассказом что она тоже служит (оказывается в театре так). Выгрузившись ночью, быстренько перекидываем все что уходит в первую заброску, я нервничаю (кажется я даже орала), потому как все очнь хаотично. Чтож, видимо для новичковой команды оптимально приезжать поездом, и времени поболе, и догнать его проще.
Заселяемся в Очаг, подьем назначен на 9.30.

\section{День -2, скальные занятия}
Утром очень-очень медленно собираемся, и выходим в сторону скальной лаборатории. Идем не по тропе(потому что увы, положиться на свою память было ошибкой), кое-как доходим до скальной лаборатории. Никого не горнит, что радует, мы с Чапой завешиваем пару коридорчиков под лазанье с верхней страховкой (как будто мы с другой можем лазить, ага), а потом и просто бросаем перила для спуска-подьема на прусах. Ребята, кажется, довольны, особо извращающиеся пытаются пролезть трассу то без одной руки, то с закрытыми глазами, то еще как. Чапа отрабатывает звание "мэстного", и рассказывает про каждую наблюдаемую вершину. Возвращаемся все же по тропе, с ужасом наблюдаем последствия закрытия границ- куча матрасников с тещами, детьми и внидорожниками оккупировали всю поляну. Сердце болью отзывается на происходящее в родном Адылсу. Вечером гуляем с Чапой до некантуя, видим в ложбине селя покореженные бочки и чью-то машину. Незнаю почему, но когда на это смотришь, становиться очень тревожно внутри.
Вечером в лагере наблюдаем как Вера "тонко" намекает Максу на массаж.

\section{День -1,в который ребята впервые по крупному меня расстраивают, и  снег, который уже лед}



По плану в этот день у нас должны быть снежные или ледовые занятия на Джанкуате. Поскольку идти до него далеко не десять минут, подьем был назначен ранний, аж на 5.00 (на самом деле я просто очень люблю рано вставать). Просыпаюсь, начинаю поглядывать на карту, дабы освежить воспоминания о долине, движения в палатках ребят нет. С интересом жду, когда же они вылезут. Проходить 10, 15, 20 минут, движения все еще нет ни в одной из трех палаток. После 25 минут мне уже не смешно и неинтересно, подхожу к каждой палатке и сообщаю, что выход через час, и успеют ли они поесть и собрать рюкзаки,   меня совершенно не волнует. \textit{Честно говоря, с таким я сталкиваюсь впервые, чтоб три палатки не смогли встать вовремя. Нет, я встречала людей которые выключали будильники со словами еще 5 минут, но обычно это 1-2 долбаеба, а никак не 9. На мой логичный вопрос "почему так?" часть людей просто стыдливо молчит, часть говорит что они не увидели движения у других и решили досыпать.} 

Утренняя возня продолжается- кто-то начинает собираться и пытается быстро приготовить завтрак, кто-то тупит и тратит драгоценное время, кто-то ходит с недовольным видом и еще претензии высказывает, типа, че не разбудили.

\textit{Я охуеваю, если честно- вам предлагают за бесплатно, с минимальным вложением сил развлекаельную программу, а все равно в ответ тебе говно прилетает. Обидно.}  

Все же, несмотря на то что КВ установленное утром выходит, едим, правда какао я откладываю на вечер, и выходим. Доходим до открытого Джанкуата, там выясняется что Кристина не взяла ветрозащиту, Чапа благородно отдает свою теплую куртку. Занимаемся, погода мерзкая, да и желания, если честно, развлекать людей после утреннего подьема нет желания. Поэтому немного позанимавшись, идем обратно. Чапа сильно помогает, когда выбирает пути отхода, и опять развлекает всех географическими справками. 

Спускаемся на Зеленую гостинницу, большая часть толпы идет смотреть на Башкаринские озера и ледник Башкара. Красиво.
На обратном пути замечаем, что крутые рантованные скарпы Кристины ведут себя совершенно неподобающим образом- отклеивается подошва аж на двух ботинках! Я заявляю что это путевка в Терскол, Кристина позже признается, что по началу испугалась- я не сразу уточнила что такое этот ваш Терскол, и она думала что я ее домой сливаю. С третьей попытки мне удается вызвать такси до Джана, Я, Кристина и два Антона едем в Терскол. План такой- купить новые боты, или в крайнем случае их арендовать. В итоге по совету ребят из Альпухи едем до Культур-Мультура, места где продается и арендовывается разная снаряга. Всего за 4000 Кристина приобретает мягкие кошаки и боты. Успех, не иначе. Потом мы перекусываем в кафе "семерка", покупаем хычины, пиво и прочие заказанные ништяки. Доезжаем до Джана, нам рады, пиво вкусное, хычины тоже. Кстати, теперь халвичный завод не выпускает ни "Терек", ни "5642". Макс М элегантно предлагает Вере "сбахнуть вальс", вера недолго ломаясь, соглашается (видимо не зря массаж делал). Ах, молодость, аж завидно. Помимо танцев, развлекаемся нашим плейлистом- кто-то нагрузил пару сопливых песен а-ля "она была самоой лууучшей девочкооой в классеее", все почему-то решают что это Верино, и долго ее стебут. 

\section{День 0, когда мы тренируемся выгуливать собак}

Утром все очень быстро вылезают из палатки, что радует. Смотрим с Чапой на небо, которое потихоньку затягивается, вспоминаем что спуск на Кашкаташ мерзкий, и решаем просто прогуляться до его верховьев по тропе(понимаю что к нам это относится так же, как трудолюбие к Рогозину, но все же, потом мы узнали что на Кашкаташе человека камнем убило, и внутренне от решения отказаться от занятий стало легче). С нами увязалась собака Ахии, и периодически мы видели или слышали ее из каких-то удаленных кустов. Немного прошли и начала портиться погода. Торчать в дождевом облаке нам с Лешей не очень хотелось, а поэтому быстренько гоним группу в низ. ПОсле обеда желающие идут лазить на НЕкантуй. Для Кирилла это чуть не оказалось фатально- после того как ребята пролезли самые простые трассы, мне пришла идея в голову оттянуть веревку на соседнюю трассу за висевшею цепочку. И все бы ничего, если бы это не делал Кирилл- неуспев закончить начатое, его три раза кусают полосатые летающие твари! Как назло, именно у него единственного из всей команды была аллергия на это. К счастью, обошлось только опухшей рукой.

Вечером договариваюсь за завтрашние машины до Тырныауза. Половина команды моется в лужах, половина в душе Ахии. Тут есть приятные новшества- теперь в душе есть нормальный смеситель, и температура воды не бинарна (раньше был формат кипяток или лед).

\textit{Тем не менее, ворчание одной Шапокляк не заставило себя ждать, де, вода чуть теплая, а не горячая.} 

\section{День 1, жаркое солнце, бывшие схроны бандитов и дедушка-травник}

Утром запаковались в 3 приоры, и поехали к точке старта. Таксист напугал меня напутствием "аккуратнее, и ни с кем не разговаривайте тут". 

\textit{Нет, мне было понятно что в 90-е тут скорее всего было стремно(взять хотя-бы рассказ Миши Расторгуенва, как его команду тут человек с автоматом ограбил), но несколько свежих отчетов меня успокаивали. Но после слов таксиста, я в каждой дырке стала видеть схрон ваххабитов.}

 Впрочем, вернемся от моих бесконечных страхов к маршруту. По пути я посмотрела на облака, вспомнила что по сокращенному варианту пути в дождь, судя по описанию, идти не очень, а поэтому пошли мы в итоге вот так:
%картинка трека


К сожалению, облака рассеялись, и мы очень долго шли по длинной грунтовой дороге, изнывая от жары и жажды- в этой долине проблемы с водой в низовьях, и дважды нас спасали путники- один раз я просила воды у остановившегося автомобилиста, а другая встреча, пожалуй, заслуживает отдельного абзаца.

Шли к тому моменту мы несколько часов, те скудные запасы воды что были уже иссякали, и тут, словно чудо господне, нам повстречался дедок, собирающий травки. Возле него мы сели покурить, а старец развлекал  нас рассказами про свое альпинисткое прошлое- рассказал как давеча он с "такими же старперами" решил сходить на Эльбрус. И что пока шел он, и что когда спустился, думал лишь об одном: "Нахераа? Только время и силы потерял". Попутно мужичок проконсультировал меня по маршруту, дал свой телефончик, и сказал где ближайший источник воды.

Распрощавшись с дедком, пошли дальше. Остановились у развилки, и пока самая лосиная часть команды пошла искать этот источник, наслаждались сиестой. Ребята ходили долго, выяснилось что источник находится в 2х километрах от нас. Ребята (героям-слава) сходили за водой для всей команды, и мы двинулись дальше. Солнце уже не так жарило, и идти было сильно приятнее. Вскоре мы вышли к запасному месту ночевки. Картина там такая: две шахты, из которых дует зловещий холоднючий воздух, из одной течет вода, которую и пьют всяк разбившие бивуак тут. Нет, честно, я была совершенно не против встать и тут, но ребята сказали что им норм, и они хотят идти еще дальше. 

Мы и пошли- сквозь туман, в основном по грунтовкам, подвергаясь нападению мычащих (сбылись мои худшие кошмары, из тумана вышел бычок, защищающий территорию). Периодически, нам даже показывали горы. До места ночевки оставалось немного, силы группы были на пределе, и это понятно- к тому моменту мы прошли чуть меньше 20 км, и набрали почти 1900 м.
Бравый Антон Синицин спас положение, и сам сбегал на разведку- и нашел чудесные места для ночевок, рядом с двумя небольшими лужами. Ребята-герои, всеми очень горжусь.
Дальше все как обычно после тяжелого дня, помогаем готовить дежурным и выманиваем людей из спальников ликером "Самбука".

Честно говоря, долина мне даже понравилась. Второй раз туда только извращенцы сунутся, но в первый даже прикольно полюбоваться шахтами и началом баксанской долины- с такого ракурса не везде получится ее увидеть.
   
\section{День 2, первая отсидка}

Утром учу Веру жечь мусор, у нее даже получается, и переодически за камнем раздается "Ничееесиии". В итоге, прособиравшись часа 2, выходим к нашему первому перевалу- Тырныауз 1А. Перевал оказался не сложным, правда на пути к нему пару раз роняем Веру. На перевал зашли в сопровождении мерзкого облака.

 \textit{Нет, я понимаю что у новичков вызывает культурный шок нахождение в облаке, я даже понимаю восторги от первых перевалов, какими бы лайтовыми они бы не были. Но у некоторых бывает особый талант доебаться. Кароче, стоим мы в этом мерзком холодном облаке на перевале, я командую быстренько сбежаться в кучку и зафотаться в любую сторону (все равно разницы нет- белое туда, белое сюда), казалось бы, сфоткался и вали, но нет, наш талантливый завхоз схватив фотоаппарат начинает выстраивать композицию из группы и локальной возвышенности на перевале. Когда я обяъвляю что времени на такую херню нет, мне обиженно заявляют, что "мне не нравиться что ты все мои хорошии инициативы обрубаешь". То есть свои дела не делаем, а подоебываться успеваем. } 
 
  На перевале Вера достает свой сюрприз- шоколадку в виде медальки, которая как бы награждает за первый перевал (у большинства это действительно первый перевал) и черт возьми, более милого сюрприза я не видела. 

Спуск с перевала- типичная приятная сыпушка типа лифт. \textit{И между прочим, те, кто потом вопил что им техники не хватило, шли не слишком уверенно даже там.} Очень быстро спускаемся в долину, с нашего следующего перевала ползет облако. Приваливаемся на обед, в конце него начинает накапывать дождь. Ставим тенты от палаток, и очень не зря- сидели мы в итоге в грозовом облаке пару часов. Дождь переодически утихал, но ненадолго, надежды пойти дальше уничтожались, и мы начинаем искать место под палатки. Саша вызывается сходить на разведку. "Почему бы и нет", подумала я. "Как интересно выглядят низовья долины", возможно подумал Саша. Спустя полчаса мы с Максом С успеваем окликнуть маленькую точку, стремительно удирающую вниз. В итоге, места под палатки нашлись в трех минутах хотьбы от нас, и чтоб не терять времени на сборе палаток, переносим их так как есть. Выглядит это забавно. 

\section{День 3, когда мы встречаем Месснера} 

Утром как всегда медленно собираемся, и выходим в сторону перевала Суарык. Дошли вполне быстро, чуть больше часа ЧХВ. Перевал очень простой, зато 1А, что нам и надо было. Красивые виды, виден даже Джаловчат (наш первый 1Б) и Эльбрус. Скушали шоколадку, пофоткались и вот это вот все. Долина спуска редкопосещаемая, и совсем набитой тропы там нет, поэтому мы промахиваемся, и траверсим сначала траву, а потом и еловый лес. Поскольку спускались мы в Верхний Баксан, я успела всем уши прожжужать на счет всяких кабаков и прочего. К сожалению, выяснилось что последний там закрылся в 18 году, но хотя бы магазины остались. 

В итоге в поселке мы были примерно в 14.00 (точное время утеряно). Пока мы ели и закупались всякими излишествами, к нам пристала собака- из тех что регулярно ходят на Сылтран с группами добродушных туристов. Неизвестно почему, этот пес решил что и мы идем что-то приятное и 
    
   
\end{document}   